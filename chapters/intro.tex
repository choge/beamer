% page 2
\frame{
    \frametitle{おさらい}
    カーネル法:
    $y(\mathbf{x}) =
    \mathbf{w}^{\mathrm{T}} {\boldsymbol \phi}(\mathbf{x})$
    のような非線形の写像(基底関数)
    $\boldsymbol \phi$を用いた線形結合を考えるときに、
    カーネル関数
    \begin{align}
    k(\mathbf{x}, \mathbf{x'}) 
    = {\boldsymbol \phi}(\mathbf{x})^{\mathrm{T}}
    {\boldsymbol \phi} (\mathbf{x'})  \tag{6.1}
    \end{align}
    を計算することで、
    $\boldsymbol \phi$による写像を直接計算せずにすむという方法。\\
    (実際の関数は、
    $k(\mathbf{x}, \mathbf{x'}) 
    = \exp( - \| \mathbf{x} - \mathbf{x'} \|^2
    / 2 \sigma^2)$
    みたいな感じ) \vspace{0.2in}

    今回は・・・\\
    これを確率的識別モデルに適用してみる。
}

% page 3
\frame{
    \frametitle{確率過程って?}
    \textbf{確率過程}{\optima(stochastic process)}とは、任意の有限な値集合
        $y(\mathbf{x}_1), \ldots, y(\mathbf{x}_N)$
    に対して、矛盾のない同時分布を与えるもの。\vspace{0.2in}

    → たとえば、時間とともに変化していく確率変数。\\
    確率変数($y(\mathbf{x})$とか)がたくさんあって、それを全部まとめたものが
    確率過程。その中から適当に$N$個を取り出したときの確率分布を考える。\\
    実生活で目にする(?)ものでいうと、ブラウン運動などがある。\vspace{0.2in}

    \textbf{ガウス過程}{\optima(Gaussian process)}は、確率過程の中でも
    たくさんの$y(\mathbf{x})$の同時分布$p(y(\mathbf{x}_1), \ldots, y(\mathbf{x}_N))$
    がガウス分布に従うようなもののこと。
}
