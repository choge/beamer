% page 21
\frame{
    \frametitle{6.4.3 超パラメータの学習}
    ガウス過程による予測は、共分散を決める際の超パラメータに依存している。
    実際の応用では、あらかじめ超パラメータを決めておくよりも、これを学習
    させることが多い。

    超パラメータの学習は、尤度関数$p(\bm{\mathsf{t}} | {\boldsymbol \theta})$
    に基づいて行われることが多い。最も単純なアプローチは、対数尤度関数
    $\ln p(\bm{\mathsf{t}} | {\boldsymbol \theta})$を最大化するような
    $\boldsymbol \theta$を求めるという方法である。\vspace{0.2in}

    ガウス過程における対数尤度関数は、
    \begin{gather}
    \ln p(\bm{\mathsf{t}} | {\boldsymbol \theta}) 
        = - \frac{1}{2} \ln | \mathbf{C}_N |
        - \frac{1}{2} \bm{\mathsf{t}}^\mathrm{T} \mathbf{C}_N^{-1} \bm{\mathsf{t}}
        - \frac{N}{2} \ln (2 \pi)
    \tag{6.69}
    \end{gather}
    で与えられる。
}

% page 22
\frame{
    \frametitle{対数尤度関数の最大化}

    先ほどの続きで、パラメータベクトル$\boldsymbol \theta$の勾配も求める。
    \begin{align}
    \frac{\partial}{\partial x} (\mathbf{A}^{-1})
       &= - \mathbf{A}^{-1} 
            \frac{\partial \mathbf{A}}{\partial x}
            \mathbf{A}^{-1}
    \tag{C.21} \\
    \frac{\partial}{\partial x} \ln |\mathbf{A}|
       &= \mathrm{Tr} \left(
               \mathbf{A}^{-1} \frac{\partial \mathbf{A}}{\partial x}
               \right)
       \tag{C.22}
    \end{align}
    を用いて(付録C参照)、
    \begin{gather}
    \frac{\partial}{\partial \theta_i} \ln p(\bm{\mathsf{t}} | {\boldsymbol \theta})
        = - \frac{1}{2} \mathrm{Tr} \left(
                \mathbf{C}_N^{-1} \frac{\partial \mathbf{C}_N}{\partial \theta_i}
                \right)
        + \frac{1}{2} \bm{\mathsf{t}}^\mathrm{T} \mathbf{C}_N^{-1}
          \frac{\partial \mathbf{C}_N}{\partial \theta_i}
          \mathbf{C}_N^{-1} \bm{\mathsf{t}}
    \tag{6.70}
    \end{gather}
    である。ただし、$\ln p(\bm{\mathsf{t}} | {\boldsymbol \theta})$は非凸関数なので
    極大点は複数あり得る。
}

\frame{
    \frametitle{6.4.3 まとめ}
    \begin{itemize}
    \item ガウス過程に現れる超パラメータの学習は、対数尤度の最大化という
        一般的な手法が用いられることが多い。
    \item ただし対数尤度関数はカーネル関数のせいで非線形なことが多いので、
        そうした場合は超パラメータの勾配も求める必要がある。
    \item ベイズ的な手法を用いて予測することもあるが、その際は近似を用いなければ
        ならない($p({\boldsymbol \theta})$や$p(\bm{\mathsf{t}}|{\boldsymbol \theta})$
        を厳密に周辺化することができない)。
    \end{itemize}
}
